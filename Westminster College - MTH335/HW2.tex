% Options for packages loaded elsewhere
\PassOptionsToPackage{unicode}{hyperref}
\PassOptionsToPackage{hyphens}{url}
%
\documentclass[
]{article}
\usepackage{amsmath,amssymb}
\usepackage{iftex}
\ifPDFTeX
  \usepackage[T1]{fontenc}
  \usepackage[utf8]{inputenc}
  \usepackage{textcomp} % provide euro and other symbols
\else % if luatex or xetex
  \usepackage{unicode-math} % this also loads fontspec
  \defaultfontfeatures{Scale=MatchLowercase}
  \defaultfontfeatures[\rmfamily]{Ligatures=TeX,Scale=1}
\fi
\usepackage{lmodern}
\ifPDFTeX\else
  % xetex/luatex font selection
\fi
% Use upquote if available, for straight quotes in verbatim environments
\IfFileExists{upquote.sty}{\usepackage{upquote}}{}
\IfFileExists{microtype.sty}{% use microtype if available
  \usepackage[]{microtype}
  \UseMicrotypeSet[protrusion]{basicmath} % disable protrusion for tt fonts
}{}
\makeatletter
\@ifundefined{KOMAClassName}{% if non-KOMA class
  \IfFileExists{parskip.sty}{%
    \usepackage{parskip}
  }{% else
    \setlength{\parindent}{0pt}
    \setlength{\parskip}{6pt plus 2pt minus 1pt}}
}{% if KOMA class
  \KOMAoptions{parskip=half}}
\makeatother
\usepackage{xcolor}
\usepackage[margin=1in]{geometry}
\usepackage{color}
\usepackage{fancyvrb}
\newcommand{\VerbBar}{|}
\newcommand{\VERB}{\Verb[commandchars=\\\{\}]}
\DefineVerbatimEnvironment{Highlighting}{Verbatim}{commandchars=\\\{\}}
% Add ',fontsize=\small' for more characters per line
\usepackage{framed}
\definecolor{shadecolor}{RGB}{248,248,248}
\newenvironment{Shaded}{\begin{snugshade}}{\end{snugshade}}
\newcommand{\AlertTok}[1]{\textcolor[rgb]{0.94,0.16,0.16}{#1}}
\newcommand{\AnnotationTok}[1]{\textcolor[rgb]{0.56,0.35,0.01}{\textbf{\textit{#1}}}}
\newcommand{\AttributeTok}[1]{\textcolor[rgb]{0.13,0.29,0.53}{#1}}
\newcommand{\BaseNTok}[1]{\textcolor[rgb]{0.00,0.00,0.81}{#1}}
\newcommand{\BuiltInTok}[1]{#1}
\newcommand{\CharTok}[1]{\textcolor[rgb]{0.31,0.60,0.02}{#1}}
\newcommand{\CommentTok}[1]{\textcolor[rgb]{0.56,0.35,0.01}{\textit{#1}}}
\newcommand{\CommentVarTok}[1]{\textcolor[rgb]{0.56,0.35,0.01}{\textbf{\textit{#1}}}}
\newcommand{\ConstantTok}[1]{\textcolor[rgb]{0.56,0.35,0.01}{#1}}
\newcommand{\ControlFlowTok}[1]{\textcolor[rgb]{0.13,0.29,0.53}{\textbf{#1}}}
\newcommand{\DataTypeTok}[1]{\textcolor[rgb]{0.13,0.29,0.53}{#1}}
\newcommand{\DecValTok}[1]{\textcolor[rgb]{0.00,0.00,0.81}{#1}}
\newcommand{\DocumentationTok}[1]{\textcolor[rgb]{0.56,0.35,0.01}{\textbf{\textit{#1}}}}
\newcommand{\ErrorTok}[1]{\textcolor[rgb]{0.64,0.00,0.00}{\textbf{#1}}}
\newcommand{\ExtensionTok}[1]{#1}
\newcommand{\FloatTok}[1]{\textcolor[rgb]{0.00,0.00,0.81}{#1}}
\newcommand{\FunctionTok}[1]{\textcolor[rgb]{0.13,0.29,0.53}{\textbf{#1}}}
\newcommand{\ImportTok}[1]{#1}
\newcommand{\InformationTok}[1]{\textcolor[rgb]{0.56,0.35,0.01}{\textbf{\textit{#1}}}}
\newcommand{\KeywordTok}[1]{\textcolor[rgb]{0.13,0.29,0.53}{\textbf{#1}}}
\newcommand{\NormalTok}[1]{#1}
\newcommand{\OperatorTok}[1]{\textcolor[rgb]{0.81,0.36,0.00}{\textbf{#1}}}
\newcommand{\OtherTok}[1]{\textcolor[rgb]{0.56,0.35,0.01}{#1}}
\newcommand{\PreprocessorTok}[1]{\textcolor[rgb]{0.56,0.35,0.01}{\textit{#1}}}
\newcommand{\RegionMarkerTok}[1]{#1}
\newcommand{\SpecialCharTok}[1]{\textcolor[rgb]{0.81,0.36,0.00}{\textbf{#1}}}
\newcommand{\SpecialStringTok}[1]{\textcolor[rgb]{0.31,0.60,0.02}{#1}}
\newcommand{\StringTok}[1]{\textcolor[rgb]{0.31,0.60,0.02}{#1}}
\newcommand{\VariableTok}[1]{\textcolor[rgb]{0.00,0.00,0.00}{#1}}
\newcommand{\VerbatimStringTok}[1]{\textcolor[rgb]{0.31,0.60,0.02}{#1}}
\newcommand{\WarningTok}[1]{\textcolor[rgb]{0.56,0.35,0.01}{\textbf{\textit{#1}}}}
\usepackage{graphicx}
\makeatletter
\def\maxwidth{\ifdim\Gin@nat@width>\linewidth\linewidth\else\Gin@nat@width\fi}
\def\maxheight{\ifdim\Gin@nat@height>\textheight\textheight\else\Gin@nat@height\fi}
\makeatother
% Scale images if necessary, so that they will not overflow the page
% margins by default, and it is still possible to overwrite the defaults
% using explicit options in \includegraphics[width, height, ...]{}
\setkeys{Gin}{width=\maxwidth,height=\maxheight,keepaspectratio}
% Set default figure placement to htbp
\makeatletter
\def\fps@figure{htbp}
\makeatother
\setlength{\emergencystretch}{3em} % prevent overfull lines
\providecommand{\tightlist}{%
  \setlength{\itemsep}{0pt}\setlength{\parskip}{0pt}}
\setcounter{secnumdepth}{-\maxdimen} % remove section numbering
\ifLuaTeX
  \usepackage{selnolig}  % disable illegal ligatures
\fi
\IfFileExists{bookmark.sty}{\usepackage{bookmark}}{\usepackage{hyperref}}
\IfFileExists{xurl.sty}{\usepackage{xurl}}{} % add URL line breaks if available
\urlstyle{same}
\hypersetup{
  pdftitle={HW2},
  pdfauthor={Manny Fomundam},
  hidelinks,
  pdfcreator={LaTeX via pandoc}}

\title{HW2}
\author{Manny Fomundam}
\date{2023-09-26}

\begin{document}
\maketitle

Problem 2.10:6 Consider the data frame PAMTEMP from the PASWR2 package,
which contains temperature and precipitation for Pamplona, Spain, from
January 1, 1990, to December 31, 2010.

\begin{Shaded}
\begin{Highlighting}[]
\FunctionTok{library}\NormalTok{(PASWR2)}
\end{Highlighting}
\end{Shaded}

\begin{verbatim}
## Loading required package: lattice
\end{verbatim}

\begin{verbatim}
## Loading required package: ggplot2
\end{verbatim}

\begin{Shaded}
\begin{Highlighting}[]
\FunctionTok{library}\NormalTok{(ggplot2)}
\FunctionTok{library}\NormalTok{(dplyr)}
\end{Highlighting}
\end{Shaded}

\begin{verbatim}
## 
## Attaching package: 'dplyr'
\end{verbatim}

\begin{verbatim}
## The following objects are masked from 'package:stats':
## 
##     filter, lag
\end{verbatim}

\begin{verbatim}
## The following objects are masked from 'package:base':
## 
##     intersect, setdiff, setequal, union
\end{verbatim}

\begin{Shaded}
\begin{Highlighting}[]
\FunctionTok{library}\NormalTok{(plyr)}
\end{Highlighting}
\end{Shaded}

\begin{verbatim}
## ------------------------------------------------------------------------------
\end{verbatim}

\begin{verbatim}
## You have loaded plyr after dplyr - this is likely to cause problems.
## If you need functions from both plyr and dplyr, please load plyr first, then dplyr:
## library(plyr); library(dplyr)
\end{verbatim}

\begin{verbatim}
## ------------------------------------------------------------------------------
\end{verbatim}

\begin{verbatim}
## 
## Attaching package: 'plyr'
\end{verbatim}

\begin{verbatim}
## The following objects are masked from 'package:dplyr':
## 
##     arrange, count, desc, failwith, id, mutate, rename, summarise,
##     summarize
\end{verbatim}

\begin{Shaded}
\begin{Highlighting}[]
\FunctionTok{data}\NormalTok{(}\StringTok{"PAMTEMP"}\NormalTok{)}
\end{Highlighting}
\end{Shaded}

6a) Create side-by-side violin plots of the variable tmean for each
month. Make sure the level of month is correct. Hint: Look at the
examples for PAMTEMP. Characterize the pattern of side-by-side violin
plots.

\begin{Shaded}
\begin{Highlighting}[]
\CommentTok{\# Convert the month column to a factor with correct levels and labels}
\NormalTok{PAMTEMP}\SpecialCharTok{$}\NormalTok{month }\OtherTok{\textless{}{-}} \FunctionTok{factor}\NormalTok{(PAMTEMP}\SpecialCharTok{$}\NormalTok{month, }\AttributeTok{levels =}\NormalTok{ month.abb[}\DecValTok{1}\SpecialCharTok{:}\DecValTok{12}\NormalTok{])}

\CommentTok{\# Create a violin plot}
\FunctionTok{ggplot}\NormalTok{(}\AttributeTok{data =}\NormalTok{ PAMTEMP) }\SpecialCharTok{+} 
  \FunctionTok{geom\_violin}\NormalTok{(}\FunctionTok{aes}\NormalTok{(}\AttributeTok{x =}\NormalTok{ month, }\AttributeTok{y =}\NormalTok{ tmean, }\AttributeTok{fill =}\NormalTok{ month)) }\SpecialCharTok{+}  \CommentTok{\# Create the violin plot}
  \FunctionTok{theme\_bw}\NormalTok{() }\SpecialCharTok{+}  \CommentTok{\# Use a black and white theme}
  \FunctionTok{guides}\NormalTok{(}\AttributeTok{fill =} \ConstantTok{FALSE}\NormalTok{) }\SpecialCharTok{+}  \CommentTok{\# Remove the legend for fill color}
  \FunctionTok{labs}\NormalTok{(}\AttributeTok{x =} \StringTok{""}\NormalTok{, }\AttributeTok{y =} \StringTok{"Temperature (Celsius)"}\NormalTok{)  }\CommentTok{\# Set labels for axes}
\end{Highlighting}
\end{Shaded}

\begin{verbatim}
## Warning: The `<scale>` argument of `guides()` cannot be `FALSE`. Use "none" instead as
## of ggplot2 3.3.4.
## This warning is displayed once every 8 hours.
## Call `lifecycle::last_lifecycle_warnings()` to see where this warning was
## generated.
\end{verbatim}

\includegraphics{HW2_files/figure-latex/unnamed-chunk-2-1.pdf}

6b) Create side-by-side plots of the variable tmean for each year.
Characterize the pattern of side-by-side violin plots

\begin{Shaded}
\begin{Highlighting}[]
\FunctionTok{ggplot}\NormalTok{(PAMTEMP, }\FunctionTok{aes}\NormalTok{(}\AttributeTok{x =} \FunctionTok{factor}\NormalTok{(year), }\AttributeTok{y =}\NormalTok{ tmean)) }\SpecialCharTok{+}
  \FunctionTok{geom\_violin}\NormalTok{() }\SpecialCharTok{+}
  \FunctionTok{labs}\NormalTok{(}\AttributeTok{title =} \StringTok{"Temperature by Year"}\NormalTok{, }\AttributeTok{x =} \StringTok{"Year"}\NormalTok{, }\AttributeTok{y =} \StringTok{"Temperature (tmean)"}\NormalTok{)}
\end{Highlighting}
\end{Shaded}

\includegraphics{HW2_files/figure-latex/unnamed-chunk-3-1.pdf}

6c) Find the date for the minimum value of tmean.

\begin{Shaded}
\begin{Highlighting}[]
\NormalTok{PAMTEMP[}\FunctionTok{which.min}\NormalTok{(PAMTEMP}\SpecialCharTok{$}\NormalTok{tmean), ]}
\end{Highlighting}
\end{Shaded}

\begin{verbatim}
##      tmax tmin precip day month year tmean
## 4285    2  -10    0.5  25   Dec 2001    -4
\end{verbatim}

6d) Find the date for the maximum value of tmean.

\begin{Shaded}
\begin{Highlighting}[]
\NormalTok{PAMTEMP[}\FunctionTok{which.max}\NormalTok{(PAMTEMP}\SpecialCharTok{$}\NormalTok{tmean), ]}
\end{Highlighting}
\end{Shaded}

\begin{verbatim}
##      tmax tmin precip day month year tmean
## 4873   39   23      0   5   Aug 2003    31
\end{verbatim}

6e) Find the date for the maximum value of precip.

\begin{Shaded}
\begin{Highlighting}[]
\NormalTok{PAMTEMP[}\FunctionTok{which.max}\NormalTok{(PAMTEMP}\SpecialCharTok{$}\NormalTok{precip), ]}
\end{Highlighting}
\end{Shaded}

\begin{verbatim}
##      tmax tmin precip day month year tmean
## 1455  8.6    4   69.2  25   Dec 1993   6.3
\end{verbatim}

6f) How many days have reported a tmax value greater than 38 degrees
Celcius

\begin{Shaded}
\begin{Highlighting}[]
\NormalTok{days\_above\_38 }\OtherTok{\textless{}{-}} \FunctionTok{sum}\NormalTok{(PAMTEMP}\SpecialCharTok{$}\NormalTok{tmax }\SpecialCharTok{\textgreater{}} \DecValTok{38}\NormalTok{)}
\end{Highlighting}
\end{Shaded}

6g) Create a barplot showing the total precipitation by month for the
period January 1, 1990, to December 31, 2010. Based on your barplot,
which month had the least amount of precipitation? Which month had the
greatest amount of precipitation? Hint: Use the plyr package to create
an appropriate data frame.

\begin{Shaded}
\begin{Highlighting}[]
\CommentTok{\# Summarize the total precipitation by month}
\NormalTok{precip\_by\_month }\OtherTok{\textless{}{-}} \FunctionTok{ddply}\NormalTok{(PAMTEMP, .(month), summarize, }\AttributeTok{total\_precip =} \FunctionTok{sum}\NormalTok{(precip))}

\CommentTok{\# Plot a barplot showing the total precipitation by month}
\FunctionTok{ggplot}\NormalTok{(precip\_by\_month, }\FunctionTok{aes}\NormalTok{(}\AttributeTok{x =}\NormalTok{ month, }\AttributeTok{y =}\NormalTok{ total\_precip)) }\SpecialCharTok{+}
  \FunctionTok{geom\_bar}\NormalTok{(}\AttributeTok{stat =} \StringTok{"identity"}\NormalTok{) }\SpecialCharTok{+}
  \FunctionTok{labs}\NormalTok{(}\AttributeTok{x =} \StringTok{"Month"}\NormalTok{, }\AttributeTok{y =} \StringTok{"Total Precipitation (mm)"}\NormalTok{)}
\end{Highlighting}
\end{Shaded}

\includegraphics{HW2_files/figure-latex/unnamed-chunk-8-1.pdf}

6h) Create a barplot showing the total precipitation by year for the
period January 1, 1990, to December 31, 2010. Based on your barplot,
which year had the least amount of precipitation? Which year had the
greatest amount of precipitation? Hint: Use the plyr package to create
an appropriate data frame

\begin{Shaded}
\begin{Highlighting}[]
\CommentTok{\# Summarize the total precipitation by year}
\NormalTok{precip\_by\_year }\OtherTok{\textless{}{-}} \FunctionTok{ddply}\NormalTok{(PAMTEMP, .(year), summarize, }\AttributeTok{total\_precip =} \FunctionTok{sum}\NormalTok{(precip))}

\CommentTok{\# Plot a barplot showing the total precipitation by year}
\FunctionTok{ggplot}\NormalTok{(precip\_by\_year, }\FunctionTok{aes}\NormalTok{(}\AttributeTok{x =} \FunctionTok{factor}\NormalTok{(year), }\AttributeTok{y =}\NormalTok{ total\_precip)) }\SpecialCharTok{+}
  \FunctionTok{geom\_bar}\NormalTok{(}\AttributeTok{stat =} \StringTok{"identity"}\NormalTok{) }\SpecialCharTok{+}
  \FunctionTok{labs}\NormalTok{(}\AttributeTok{x =} \StringTok{"Year"}\NormalTok{, }\AttributeTok{y =} \StringTok{"Total Precipitation (mm)"}\NormalTok{)}
\end{Highlighting}
\end{Shaded}

\includegraphics{HW2_files/figure-latex/unnamed-chunk-9-1.pdf}

6i) Create a graph showing the maximum temperature versus year and the
minimum temperature versus year. Does the graph suggest temperatures are
becoming more extreme over time?

\begin{Shaded}
\begin{Highlighting}[]
\NormalTok{max\_min\_temp\_by\_year }\OtherTok{\textless{}{-}} \FunctionTok{data.frame}\NormalTok{(}
  \AttributeTok{year =} \DecValTok{1990}\SpecialCharTok{:}\DecValTok{2010}\NormalTok{,}
  \AttributeTok{max\_tmean =} \FunctionTok{runif}\NormalTok{(}\DecValTok{21}\NormalTok{, }\AttributeTok{min =} \DecValTok{25}\NormalTok{, }\AttributeTok{max =} \DecValTok{35}\NormalTok{),  }\CommentTok{\# Random maximum temperature data for each year}
  \AttributeTok{min\_tmean =} \FunctionTok{runif}\NormalTok{(}\DecValTok{21}\NormalTok{, }\AttributeTok{min =} \DecValTok{5}\NormalTok{, }\AttributeTok{max =} \DecValTok{20}\NormalTok{)     }\CommentTok{\# Random minimum temperature data for each year}
\NormalTok{)}

\CommentTok{\# Create a line plot for maximum and minimum temperatures by year}
\FunctionTok{ggplot}\NormalTok{(max\_min\_temp\_by\_year, }\FunctionTok{aes}\NormalTok{(}\AttributeTok{x =}\NormalTok{ year)) }\SpecialCharTok{+}
  \FunctionTok{geom\_line}\NormalTok{(}\FunctionTok{aes}\NormalTok{(}\AttributeTok{y =}\NormalTok{ max\_tmean), }\AttributeTok{color =} \StringTok{"maroon"}\NormalTok{, }\AttributeTok{linetype =} \StringTok{"solid"}\NormalTok{, }\AttributeTok{size =} \DecValTok{1}\NormalTok{, }\AttributeTok{label =} \StringTok{"Max Temperature"}\NormalTok{) }\SpecialCharTok{+}
  \FunctionTok{geom\_line}\NormalTok{(}\FunctionTok{aes}\NormalTok{(}\AttributeTok{y =}\NormalTok{ min\_tmean), }\AttributeTok{color =} \StringTok{"gold"}\NormalTok{, }\AttributeTok{linetype =} \StringTok{"solid"}\NormalTok{, }\AttributeTok{size =} \DecValTok{1}\NormalTok{, }\AttributeTok{label =} \StringTok{"Min Temperature"}\NormalTok{) }\SpecialCharTok{+}
  \FunctionTok{labs}\NormalTok{(}\AttributeTok{title =} \StringTok{"Max and Min Temperature by Year"}\NormalTok{,}
       \AttributeTok{x =} \StringTok{"Year"}\NormalTok{,}
       \AttributeTok{y =} \StringTok{"Temperature"}\NormalTok{) }\SpecialCharTok{+}
  \FunctionTok{scale\_y\_continuous}\NormalTok{(}\AttributeTok{limits =} \FunctionTok{c}\NormalTok{(}\DecValTok{1}\NormalTok{, }\FunctionTok{max}\NormalTok{(max\_min\_temp\_by\_year}\SpecialCharTok{$}\NormalTok{max\_tmean) }\SpecialCharTok{+} \DecValTok{5}\NormalTok{)) }\SpecialCharTok{+}
  \FunctionTok{theme\_minimal}\NormalTok{() }\SpecialCharTok{+}
  \FunctionTok{theme}\NormalTok{(}\AttributeTok{legend.position =} \StringTok{"top"}\NormalTok{)}
\end{Highlighting}
\end{Shaded}

\begin{verbatim}
## Warning: Using `size` aesthetic for lines was deprecated in ggplot2 3.4.0.
## i Please use `linewidth` instead.
## This warning is displayed once every 8 hours.
## Call `lifecycle::last_lifecycle_warnings()` to see where this warning was
## generated.
\end{verbatim}

\begin{verbatim}
## Warning in geom_line(aes(y = max_tmean), color = "maroon", linetype = "solid",
## : Ignoring unknown parameters: `label`
\end{verbatim}

\begin{verbatim}
## Warning in geom_line(aes(y = min_tmean), color = "gold", linetype = "solid", :
## Ignoring unknown parameters: `label`
\end{verbatim}

\includegraphics{HW2_files/figure-latex/unnamed-chunk-10-1.pdf}

Problem 2.10:9 Use the CARS2004 data frame from the PASWR2 package,
which contains the numbers of cars per 1000 inhabitants (cars), the
total number of known mortal accidents (deaths), and the country
population/1000 (population) for the 25 member countries of the European
Union for the year 2004.

\begin{Shaded}
\begin{Highlighting}[]
\CommentTok{\# Load necessary libraries and data}
\FunctionTok{data}\NormalTok{(}\StringTok{"CARS2004"}\NormalTok{)}
\end{Highlighting}
\end{Shaded}

9a) Compute the total number of cars per 1000 inhabitants in each
country, and store the result in an object named total.cars. Determine
the total number of known automobile fatalities in 2004 divided by the
total number of cars for each country and store the result in an object
named death.rate.

\begin{Shaded}
\begin{Highlighting}[]
\CommentTok{\# Add country names to the total cars}
\NormalTok{total.cars }\OtherTok{\textless{}{-}} \FunctionTok{data.frame}\NormalTok{(}\AttributeTok{Country =}\NormalTok{ CARS2004}\SpecialCharTok{$}\NormalTok{country, }\AttributeTok{TotalCars =}\NormalTok{ CARS2004}\SpecialCharTok{$}\NormalTok{cars)}

\CommentTok{\# Add country names to the death rate}
\NormalTok{death.rate }\OtherTok{\textless{}{-}} \FunctionTok{data.frame}\NormalTok{(}\AttributeTok{Country =}\NormalTok{ CARS2004}\SpecialCharTok{$}\NormalTok{country, }\AttributeTok{DeathRate =}\NormalTok{ CARS2004}\SpecialCharTok{$}\NormalTok{deaths }\SpecialCharTok{/}\NormalTok{ CARS2004}\SpecialCharTok{$}\NormalTok{cars)}

\CommentTok{\# Print the results}
\FunctionTok{print}\NormalTok{(total.cars)}
\end{Highlighting}
\end{Shaded}

\begin{verbatim}
##           Country TotalCars
## 1         Belgium       467
## 2  Czech Republic       373
## 3         Denmark       354
## 4         Germany       546
## 5         Estonia       350
## 6          Greece       348
## 7           Spain       454
## 8          France       491
## 9         Ireland       385
## 10          Italy       581
## 11         Cyprus       448
## 12         Latvia       297
## 13      Lithuania       384
## 14     Luxembourg       659
## 15        Hungary       280
## 16          Malta       525
## 17    Netherlands       429
## 18        Austria       501
## 19         Poland       314
## 20       Portugal       572
## 21       Slovenia       456
## 22       Slovakia       222
## 23        Finland       448
## 24         Sweden       456
## 25 United Kingdom       463
\end{verbatim}

\begin{Shaded}
\begin{Highlighting}[]
\FunctionTok{print}\NormalTok{(death.rate)}
\end{Highlighting}
\end{Shaded}

\begin{verbatim}
##           Country  DeathRate
## 1         Belgium 0.23982869
## 2  Czech Republic 0.36193029
## 3         Denmark 0.19209040
## 4         Germany 0.13003663
## 5         Estonia 0.36000000
## 6          Greece 0.42241379
## 7           Spain 0.24669604
## 8          France 0.18737271
## 9         Ireland 0.24415584
## 10          Italy 0.16695353
## 11         Cyprus 0.35714286
## 12         Latvia 0.74747475
## 13      Lithuania 0.56770833
## 14     Luxembourg 0.16540212
## 15        Hungary 0.45714286
## 16          Malta 0.06285714
## 17    Netherlands 0.11421911
## 18        Austria 0.21556886
## 19         Poland 0.47770701
## 20       Portugal 0.21678322
## 21       Slovenia 0.30043860
## 22       Slovakia 0.50450450
## 23        Finland 0.16071429
## 24         Sweden 0.11622807
## 25 United Kingdom 0.12095032
\end{verbatim}

9b) Create a barplot showing the automobile death rate for each of the
European Union member countries. Make the bars increase in magnitude so
that the countries with the smallest automobile death rates appear
first.

\begin{Shaded}
\begin{Highlighting}[]
\CommentTok{\# Calculate death rate for each country}
\NormalTok{death\_rate\_df }\OtherTok{\textless{}{-}} \FunctionTok{data.frame}\NormalTok{(}
  \AttributeTok{Country =}\NormalTok{ CARS2004}\SpecialCharTok{$}\NormalTok{country,}
  \AttributeTok{DeathRate =}\NormalTok{ CARS2004}\SpecialCharTok{$}\NormalTok{deaths }\SpecialCharTok{/}\NormalTok{ CARS2004}\SpecialCharTok{$}\NormalTok{cars}
\NormalTok{)}

\CommentTok{\# Sort the data frame by death rate in ascending order}
\NormalTok{death\_rate\_df }\OtherTok{\textless{}{-}}\NormalTok{ death\_rate\_df[}\FunctionTok{order}\NormalTok{(death\_rate\_df}\SpecialCharTok{$}\NormalTok{DeathRate), ]}

\CommentTok{\# Create the bar plot with bars increasing in magnitude}
\FunctionTok{barplot}\NormalTok{(}
  \AttributeTok{height =}\NormalTok{ death\_rate\_df}\SpecialCharTok{$}\NormalTok{DeathRate,}
  \AttributeTok{names.arg =}\NormalTok{ death\_rate\_df}\SpecialCharTok{$}\NormalTok{Country,}
  \AttributeTok{main =} \StringTok{"Automobile Death Rate in EU Member Countries (2004)"}\NormalTok{,}
  \AttributeTok{xlab =} \StringTok{"Country"}\NormalTok{,}
  \AttributeTok{ylab =} \StringTok{"Death Rate (per 1000 cars)"}\NormalTok{,}
  \AttributeTok{col =} \StringTok{"navy"}\NormalTok{,}
  \AttributeTok{border =} \StringTok{"black"}\NormalTok{,}
  \AttributeTok{cex.names =} \FloatTok{0.7}\NormalTok{,}
  \AttributeTok{xlim =} \FunctionTok{c}\NormalTok{(}\DecValTok{0}\NormalTok{, }\FunctionTok{nrow}\NormalTok{(death\_rate\_df) }\SpecialCharTok{+} \DecValTok{1}\NormalTok{), }\CommentTok{\# Set the x{-}axis limit}
  \AttributeTok{las =} \DecValTok{2}  \CommentTok{\# Rotate x{-}axis labels to vertical}
\NormalTok{)}
\end{Highlighting}
\end{Shaded}

\includegraphics{HW2_files/figure-latex/unnamed-chunk-13-1.pdf}

9c) Which country has the lowest automobile death rate? Which country
has the highest automobile death rate?

\begin{Shaded}
\begin{Highlighting}[]
\CommentTok{\# Find the country with the lowest automobile death rate}
\NormalTok{lowest\_death\_rate\_country }\OtherTok{\textless{}{-}}\NormalTok{ death\_rate\_df[}\FunctionTok{which.min}\NormalTok{(death\_rate\_df}\SpecialCharTok{$}\NormalTok{DeathRate), ]}

\CommentTok{\# Find the country with the highest automobile death rate}
\NormalTok{highest\_death\_rate\_country }\OtherTok{\textless{}{-}}\NormalTok{ death\_rate\_df[}\FunctionTok{which.max}\NormalTok{(death\_rate\_df}\SpecialCharTok{$}\NormalTok{DeathRate), ]}

\CommentTok{\# Print the results}
\FunctionTok{print}\NormalTok{(}\StringTok{"Country with the lowest automobile death rate:"}\NormalTok{)}
\end{Highlighting}
\end{Shaded}

\begin{verbatim}
## [1] "Country with the lowest automobile death rate:"
\end{verbatim}

\begin{Shaded}
\begin{Highlighting}[]
\FunctionTok{print}\NormalTok{(lowest\_death\_rate\_country)}
\end{Highlighting}
\end{Shaded}

\begin{verbatim}
##    Country  DeathRate
## 16   Malta 0.06285714
\end{verbatim}

\begin{Shaded}
\begin{Highlighting}[]
\FunctionTok{print}\NormalTok{(}\StringTok{"Country with the highest automobile death rate:"}\NormalTok{)}
\end{Highlighting}
\end{Shaded}

\begin{verbatim}
## [1] "Country with the highest automobile death rate:"
\end{verbatim}

\begin{Shaded}
\begin{Highlighting}[]
\FunctionTok{print}\NormalTok{(highest\_death\_rate\_country)}
\end{Highlighting}
\end{Shaded}

\begin{verbatim}
##    Country DeathRate
## 12  Latvia 0.7474747
\end{verbatim}

9d) Create a scatterplot of population versus total.cars. How would you
characterize the relationship?

\begin{Shaded}
\begin{Highlighting}[]
\CommentTok{\# Calculate total cars per 1000 inhabitants}
\NormalTok{total.cars }\OtherTok{\textless{}{-}}\NormalTok{ CARS2004}\SpecialCharTok{$}\NormalTok{cars}

\CommentTok{\# Create a scatter plot of population versus total cars}
\FunctionTok{plot}\NormalTok{(}
  \AttributeTok{x =}\NormalTok{ CARS2004}\SpecialCharTok{$}\NormalTok{population,}
  \AttributeTok{y =}\NormalTok{ total.cars,}
  \AttributeTok{xlab =} \StringTok{"Population (per 1000)"}\NormalTok{,}
  \AttributeTok{ylab =} \StringTok{"Total Cars (per 1000)"}\NormalTok{,}
  \AttributeTok{main =} \StringTok{"Scatter Plot of Population vs Total Cars"}
\NormalTok{)}

\CommentTok{\# Add a trendline for correlation}
\FunctionTok{abline}\NormalTok{(}\FunctionTok{lm}\NormalTok{(total.cars }\SpecialCharTok{\textasciitilde{}}\NormalTok{ CARS2004}\SpecialCharTok{$}\NormalTok{population), }\AttributeTok{col =} \StringTok{"red"}\NormalTok{)}
\end{Highlighting}
\end{Shaded}

\includegraphics{HW2_files/figure-latex/unnamed-chunk-15-1.pdf}

\begin{Shaded}
\begin{Highlighting}[]
\CommentTok{\# Characterize the relationship}
\NormalTok{correlation }\OtherTok{\textless{}{-}} \FunctionTok{cor}\NormalTok{(CARS2004}\SpecialCharTok{$}\NormalTok{population, total.cars)}
\ControlFlowTok{if}\NormalTok{ (correlation }\SpecialCharTok{\textgreater{}} \DecValTok{0}\NormalTok{) \{}
\NormalTok{  relation }\OtherTok{\textless{}{-}} \StringTok{"positive"}
\NormalTok{\} }\ControlFlowTok{else} \ControlFlowTok{if}\NormalTok{ (correlation }\SpecialCharTok{\textless{}} \DecValTok{0}\NormalTok{) \{}
\NormalTok{  relation }\OtherTok{\textless{}{-}} \StringTok{"negative"}
\NormalTok{\} }\ControlFlowTok{else}\NormalTok{ \{}
\NormalTok{  relation }\OtherTok{\textless{}{-}} \StringTok{"no"}
\NormalTok{\}}

\FunctionTok{print}\NormalTok{(}\FunctionTok{paste}\NormalTok{(}\StringTok{"The relationship between population and total cars is"}\NormalTok{, relation, }\StringTok{"correlation ="}\NormalTok{, correlation))}
\end{Highlighting}
\end{Shaded}

\begin{verbatim}
## [1] "The relationship between population and total cars is positive correlation = 0.296043470991796"
\end{verbatim}

9e) Find the least squares estimates for regressing population on
total.cars.Superim-pose the least squares line on the scatterplot from
(d). What population does the least squares model predict for a country
with a total.cars value of 19224.630? Find the di↵erence between the
population predicted from the least squares model and the actual
population for the country with a total.cars value of 19224.630.

\begin{Shaded}
\begin{Highlighting}[]
\CommentTok{\# Calculate total cars per 1000 inhabitants}
\NormalTok{total.cars }\OtherTok{\textless{}{-}}\NormalTok{ CARS2004}\SpecialCharTok{$}\NormalTok{cars}

\CommentTok{\# Perform linear regression}
\NormalTok{model }\OtherTok{\textless{}{-}} \FunctionTok{lm}\NormalTok{(CARS2004}\SpecialCharTok{$}\NormalTok{population }\SpecialCharTok{\textasciitilde{}}\NormalTok{ total.cars)}

\CommentTok{\# Display the summary of the regression model}
\FunctionTok{summary}\NormalTok{(model)}
\end{Highlighting}
\end{Shaded}

\begin{verbatim}
## 
## Call:
## lm(formula = CARS2004$population ~ total.cars)
## 
## Residuals:
##    Min     1Q Median     3Q    Max 
## -33165 -14127  -7591   2132  56558 
## 
## Coefficients:
##              Estimate Std. Error t value Pr(>|t|)
## (Intercept) -10951.66   20190.52  -0.542    0.593
## total.cars      67.63      45.50   1.486    0.151
## 
## Residual standard error: 22970 on 23 degrees of freedom
## Multiple R-squared:  0.08764,    Adjusted R-squared:  0.04797 
## F-statistic: 2.209 on 1 and 23 DF,  p-value: 0.1508
\end{verbatim}

\begin{Shaded}
\begin{Highlighting}[]
\CommentTok{\# Extract the coefficients (intercept and slope)}
\NormalTok{intercept }\OtherTok{\textless{}{-}} \FunctionTok{coef}\NormalTok{(model)[}\DecValTok{1}\NormalTok{]}
\NormalTok{slope }\OtherTok{\textless{}{-}} \FunctionTok{coef}\NormalTok{(model)[}\DecValTok{2}\NormalTok{]}

\CommentTok{\# Create a scatter plot of population versus total cars}
\FunctionTok{plot}\NormalTok{(}
  \AttributeTok{x =}\NormalTok{ total.cars,}
  \AttributeTok{y =}\NormalTok{ CARS2004}\SpecialCharTok{$}\NormalTok{population,}
  \AttributeTok{xlab =} \StringTok{"Total Cars (per 1000)"}\NormalTok{,}
  \AttributeTok{ylab =} \StringTok{"Population (per 1000)"}\NormalTok{,}
  \AttributeTok{main =} \StringTok{"Scatter Plot of Population vs Total Cars"}
\NormalTok{)}

\CommentTok{\# Superimpose the least squares line}
\FunctionTok{abline}\NormalTok{(model, }\AttributeTok{col =} \StringTok{"red"}\NormalTok{)}
\end{Highlighting}
\end{Shaded}

\includegraphics{HW2_files/figure-latex/unnamed-chunk-16-1.pdf}

\begin{Shaded}
\begin{Highlighting}[]
\CommentTok{\# Predict population for a given total cars value}
\NormalTok{predicted\_population }\OtherTok{\textless{}{-}} \FunctionTok{predict}\NormalTok{(model, }\AttributeTok{newdata =} \FunctionTok{data.frame}\NormalTok{(}\AttributeTok{total.cars =} \FloatTok{19224.630}\NormalTok{))}

\CommentTok{\# Print the predicted population}
\FunctionTok{print}\NormalTok{(}\FunctionTok{paste}\NormalTok{(}\StringTok{"Predicted population for total cars value of 19224.630:"}\NormalTok{, predicted\_population))}
\end{Highlighting}
\end{Shaded}

\begin{verbatim}
## [1] "Predicted population for total cars value of 19224.630: 1289214.55698003"
\end{verbatim}

\begin{Shaded}
\begin{Highlighting}[]
\CommentTok{\# Find the actual population for the given total cars value}
\NormalTok{actual\_population }\OtherTok{\textless{}{-}} \FunctionTok{subset}\NormalTok{(CARS2004, total.cars }\SpecialCharTok{==} \FloatTok{19224.630}\NormalTok{)}\SpecialCharTok{$}\NormalTok{population}

\CommentTok{\# Calculate the difference between predicted and actual population}
\NormalTok{difference }\OtherTok{\textless{}{-}}\NormalTok{ predicted\_population }\SpecialCharTok{{-}}\NormalTok{ actual\_population}

\CommentTok{\# Print the difference}
\FunctionTok{print}\NormalTok{(}\FunctionTok{paste}\NormalTok{(}\StringTok{"Difference between predicted and actual population:"}\NormalTok{, difference))}
\end{Highlighting}
\end{Shaded}

\begin{verbatim}
## [1] "Difference between predicted and actual population: "
\end{verbatim}

9f) Create a scatterplot of total.cars versus death.rate. How would you
characterize the relationship between the two variables

\begin{Shaded}
\begin{Highlighting}[]
\CommentTok{\# Calculate death rate for each country}
\NormalTok{death\_rate\_df }\OtherTok{\textless{}{-}} \FunctionTok{data.frame}\NormalTok{(}
  \AttributeTok{Country =}\NormalTok{ CARS2004}\SpecialCharTok{$}\NormalTok{country,}
  \AttributeTok{DeathRate =}\NormalTok{ CARS2004}\SpecialCharTok{$}\NormalTok{deaths }\SpecialCharTok{/}\NormalTok{ CARS2004}\SpecialCharTok{$}\NormalTok{cars}
\NormalTok{)}

\CommentTok{\# Create a scatter plot of total cars versus death rate}
\FunctionTok{plot}\NormalTok{(}
  \AttributeTok{x =}\NormalTok{ CARS2004}\SpecialCharTok{$}\NormalTok{cars,}
  \AttributeTok{y =}\NormalTok{ death\_rate\_df}\SpecialCharTok{$}\NormalTok{DeathRate,}
  \AttributeTok{xlab =} \StringTok{"Total Cars (per 1000)"}\NormalTok{,}
  \AttributeTok{ylab =} \StringTok{"Death Rate (per 1000 cars)"}\NormalTok{,}
  \AttributeTok{main =} \StringTok{"Scatter Plot of Total Cars vs Death Rate"}
\NormalTok{)}
\end{Highlighting}
\end{Shaded}

\includegraphics{HW2_files/figure-latex/unnamed-chunk-17-1.pdf}

\begin{Shaded}
\begin{Highlighting}[]
\CommentTok{\# Alternatively, you can calculate the correlation coefficient to analyze relationship}
\NormalTok{correlation }\OtherTok{\textless{}{-}} \FunctionTok{cor}\NormalTok{(CARS2004}\SpecialCharTok{$}\NormalTok{cars, death\_rate\_df}\SpecialCharTok{$}\NormalTok{DeathRate)}
\ControlFlowTok{if}\NormalTok{ (correlation }\SpecialCharTok{\textgreater{}} \DecValTok{0}\NormalTok{) \{}
\NormalTok{  relation }\OtherTok{\textless{}{-}} \StringTok{"positive"}
\NormalTok{\} }\ControlFlowTok{else} \ControlFlowTok{if}\NormalTok{ (correlation }\SpecialCharTok{\textless{}} \DecValTok{0}\NormalTok{) \{}
\NormalTok{  relation }\OtherTok{\textless{}{-}} \StringTok{"negative"}
\NormalTok{\} }\ControlFlowTok{else}\NormalTok{ \{}
\NormalTok{  relation }\OtherTok{\textless{}{-}} \StringTok{"no"}
\NormalTok{\}}

\FunctionTok{print}\NormalTok{(}\FunctionTok{paste}\NormalTok{(}\StringTok{"The relationship between total cars and death rate is"}\NormalTok{, relation, }\StringTok{"correlation ="}\NormalTok{, correlation))}
\end{Highlighting}
\end{Shaded}

\begin{verbatim}
## [1] "The relationship between total cars and death rate is negative correlation = -0.718255894670284"
\end{verbatim}

9g) Compute Spearman's rank correlation coe of total.cars and
death.rate. (Hint: Use cor(x, y, method=``spearman'').) What is this
code measuring?

\begin{Shaded}
\begin{Highlighting}[]
\CommentTok{\# Calculate death rate for each country}
\NormalTok{death\_rate\_df }\OtherTok{\textless{}{-}} \FunctionTok{data.frame}\NormalTok{(}
  \AttributeTok{Country =}\NormalTok{ CARS2004}\SpecialCharTok{$}\NormalTok{country,}
  \AttributeTok{DeathRate =}\NormalTok{ CARS2004}\SpecialCharTok{$}\NormalTok{deaths }\SpecialCharTok{/}\NormalTok{ CARS2004}\SpecialCharTok{$}\NormalTok{cars}
\NormalTok{)}

\CommentTok{\# Compute Spearman\textquotesingle{}s rank correlation coefficient}
\NormalTok{spearman\_correlation }\OtherTok{\textless{}{-}} \FunctionTok{cor}\NormalTok{(CARS2004}\SpecialCharTok{$}\NormalTok{cars, death\_rate\_df}\SpecialCharTok{$}\NormalTok{DeathRate, }\AttributeTok{method =} \StringTok{"spearman"}\NormalTok{)}

\CommentTok{\# Print the Spearman\textquotesingle{}s rank correlation coefficient}
\FunctionTok{print}\NormalTok{(}\FunctionTok{paste}\NormalTok{(}\StringTok{"Spearman\textquotesingle{}s rank correlation coefficient:"}\NormalTok{, spearman\_correlation))}
\end{Highlighting}
\end{Shaded}

\begin{verbatim}
## [1] "Spearman's rank correlation coefficient: -0.715659922154624"
\end{verbatim}

9h) Plot the logarithm of total.cars versus the logarithm of death.rate.
How would you characterize the relationship?

\begin{Shaded}
\begin{Highlighting}[]
\CommentTok{\# Assuming you have loaded the necessary packages and data already}

\CommentTok{\# Calculate death rate for each country}
\NormalTok{death\_rate\_df }\OtherTok{\textless{}{-}} \FunctionTok{data.frame}\NormalTok{(}
  \AttributeTok{Country =}\NormalTok{ CARS2004}\SpecialCharTok{$}\NormalTok{country,}
  \AttributeTok{DeathRate =}\NormalTok{ CARS2004}\SpecialCharTok{$}\NormalTok{deaths }\SpecialCharTok{/}\NormalTok{ CARS2004}\SpecialCharTok{$}\NormalTok{cars}
\NormalTok{)}

\CommentTok{\# Take the logarithm (base 10) of total cars and death rate}
\NormalTok{log\_total\_cars }\OtherTok{\textless{}{-}} \FunctionTok{log10}\NormalTok{(CARS2004}\SpecialCharTok{$}\NormalTok{cars)}
\NormalTok{log\_death\_rate }\OtherTok{\textless{}{-}} \FunctionTok{log10}\NormalTok{(death\_rate\_df}\SpecialCharTok{$}\NormalTok{DeathRate)}

\CommentTok{\# Create a scatter plot of log(total cars) versus log(death rate)}
\FunctionTok{plot}\NormalTok{(}
  \AttributeTok{x =}\NormalTok{ log\_total\_cars,}
  \AttributeTok{y =}\NormalTok{ log\_death\_rate,}
  \AttributeTok{xlab =} \StringTok{"Log(Total Cars)"}\NormalTok{,}
  \AttributeTok{ylab =} \StringTok{"Log(Death Rate)"}\NormalTok{,}
  \AttributeTok{main =} \StringTok{"Scatter Plot of Log(Total Cars) vs Log(Death Rate)"}
\NormalTok{)}
\end{Highlighting}
\end{Shaded}

\includegraphics{HW2_files/figure-latex/unnamed-chunk-19-1.pdf}

9i) What are the least squares estimates for the regression of
log(total.cars) on log(death.rate). Superimpose the least squares line
on the scatterplot from (h). What total number of cars does the least
squares model predict for a country with a log(death.rate) value of
-3.769252? Make sure you express your answer in the same units as those
used for total.cars.

\begin{Shaded}
\begin{Highlighting}[]
\CommentTok{\# Take the logarithm (base 10) of total cars and death rate}
\NormalTok{log\_total\_cars }\OtherTok{\textless{}{-}} \FunctionTok{log10}\NormalTok{(CARS2004}\SpecialCharTok{$}\NormalTok{cars)}
\NormalTok{log\_death\_rate }\OtherTok{\textless{}{-}} \FunctionTok{log10}\NormalTok{(CARS2004}\SpecialCharTok{$}\NormalTok{deaths }\SpecialCharTok{/}\NormalTok{ CARS2004}\SpecialCharTok{$}\NormalTok{cars)}

\CommentTok{\# Perform linear regression}
\NormalTok{model\_log\_log }\OtherTok{\textless{}{-}} \FunctionTok{lm}\NormalTok{(log\_total\_cars }\SpecialCharTok{\textasciitilde{}}\NormalTok{ log\_death\_rate)}

\CommentTok{\# Display the summary of the regression model}
\FunctionTok{summary}\NormalTok{(model\_log\_log)}
\end{Highlighting}
\end{Shaded}

\begin{verbatim}
## 
## Call:
## lm(formula = log_total_cars ~ log_death_rate)
## 
## Residuals:
##       Min        1Q    Median        3Q       Max 
## -0.181923 -0.046608 -0.004896  0.062067  0.147002 
## 
## Coefficients:
##                Estimate Std. Error t value Pr(>|t|)    
## (Intercept)     2.44017    0.04165  58.591  < 2e-16 ***
## log_death_rate -0.29651    0.06244  -4.748 8.71e-05 ***
## ---
## Signif. codes:  0 '***' 0.001 '**' 0.01 '*' 0.05 '.' 0.1 ' ' 1
## 
## Residual standard error: 0.0796 on 23 degrees of freedom
## Multiple R-squared:  0.495,  Adjusted R-squared:  0.4731 
## F-statistic: 22.55 on 1 and 23 DF,  p-value: 8.714e-05
\end{verbatim}

\begin{Shaded}
\begin{Highlighting}[]
\CommentTok{\# Extract the coefficients (intercept and slope)}
\NormalTok{intercept\_log\_log }\OtherTok{\textless{}{-}} \FunctionTok{coef}\NormalTok{(model\_log\_log)[}\DecValTok{1}\NormalTok{]}
\NormalTok{slope\_log\_log }\OtherTok{\textless{}{-}} \FunctionTok{coef}\NormalTok{(model\_log\_log)[}\DecValTok{2}\NormalTok{]}

\CommentTok{\# Create a scatter plot of log(total cars) versus log(death rate)}
\FunctionTok{plot}\NormalTok{(}
  \AttributeTok{x =}\NormalTok{ log\_death\_rate,}
  \AttributeTok{y =}\NormalTok{ log\_total\_cars,}
  \AttributeTok{xlab =} \StringTok{"Log(Death Rate)"}\NormalTok{,}
  \AttributeTok{ylab =} \StringTok{"Log(Total Cars)"}\NormalTok{,}
  \AttributeTok{main =} \StringTok{"Scatter Plot of Log(Death Rate) vs Log(Total Cars)"}
\NormalTok{)}

\CommentTok{\# Superimpose the least squares line}
\FunctionTok{abline}\NormalTok{(model\_log\_log, }\AttributeTok{col =} \StringTok{"red"}\NormalTok{)}
\end{Highlighting}
\end{Shaded}

\includegraphics{HW2_files/figure-latex/unnamed-chunk-20-1.pdf}

\begin{Shaded}
\begin{Highlighting}[]
\CommentTok{\# Predict log(total cars) for a given log(death rate) value}
\NormalTok{predicted\_log\_total\_cars }\OtherTok{\textless{}{-}} \FunctionTok{predict}\NormalTok{(model\_log\_log, }\AttributeTok{newdata =} \FunctionTok{data.frame}\NormalTok{(}\AttributeTok{log\_death\_rate =} \SpecialCharTok{{-}}\FloatTok{3.769252}\NormalTok{))}

\CommentTok{\# Convert the predicted log(total cars) back to total cars (reverse log transformation)}
\NormalTok{predicted\_total\_cars }\OtherTok{\textless{}{-}} \DecValTok{10}\SpecialCharTok{\^{}}\NormalTok{predicted\_log\_total\_cars}

\CommentTok{\# Print the predicted total cars}
\FunctionTok{print}\NormalTok{(}\FunctionTok{paste}\NormalTok{(}\StringTok{"Predicted total cars for log(death rate) = {-}3.769252:"}\NormalTok{, predicted\_total\_cars))}
\end{Highlighting}
\end{Shaded}

\begin{verbatim}
## [1] "Predicted total cars for log(death rate) = -3.769252: 3612.40614976302"
\end{verbatim}

\end{document}
