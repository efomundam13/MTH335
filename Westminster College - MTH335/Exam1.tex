% Options for packages loaded elsewhere
\PassOptionsToPackage{unicode}{hyperref}
\PassOptionsToPackage{hyphens}{url}
%
\documentclass[
]{article}
\usepackage{amsmath,amssymb}
\usepackage{iftex}
\ifPDFTeX
  \usepackage[T1]{fontenc}
  \usepackage[utf8]{inputenc}
  \usepackage{textcomp} % provide euro and other symbols
\else % if luatex or xetex
  \usepackage{unicode-math} % this also loads fontspec
  \defaultfontfeatures{Scale=MatchLowercase}
  \defaultfontfeatures[\rmfamily]{Ligatures=TeX,Scale=1}
\fi
\usepackage{lmodern}
\ifPDFTeX\else
  % xetex/luatex font selection
\fi
% Use upquote if available, for straight quotes in verbatim environments
\IfFileExists{upquote.sty}{\usepackage{upquote}}{}
\IfFileExists{microtype.sty}{% use microtype if available
  \usepackage[]{microtype}
  \UseMicrotypeSet[protrusion]{basicmath} % disable protrusion for tt fonts
}{}
\makeatletter
\@ifundefined{KOMAClassName}{% if non-KOMA class
  \IfFileExists{parskip.sty}{%
    \usepackage{parskip}
  }{% else
    \setlength{\parindent}{0pt}
    \setlength{\parskip}{6pt plus 2pt minus 1pt}}
}{% if KOMA class
  \KOMAoptions{parskip=half}}
\makeatother
\usepackage{xcolor}
\usepackage[margin=1in]{geometry}
\usepackage{color}
\usepackage{fancyvrb}
\newcommand{\VerbBar}{|}
\newcommand{\VERB}{\Verb[commandchars=\\\{\}]}
\DefineVerbatimEnvironment{Highlighting}{Verbatim}{commandchars=\\\{\}}
% Add ',fontsize=\small' for more characters per line
\usepackage{framed}
\definecolor{shadecolor}{RGB}{248,248,248}
\newenvironment{Shaded}{\begin{snugshade}}{\end{snugshade}}
\newcommand{\AlertTok}[1]{\textcolor[rgb]{0.94,0.16,0.16}{#1}}
\newcommand{\AnnotationTok}[1]{\textcolor[rgb]{0.56,0.35,0.01}{\textbf{\textit{#1}}}}
\newcommand{\AttributeTok}[1]{\textcolor[rgb]{0.13,0.29,0.53}{#1}}
\newcommand{\BaseNTok}[1]{\textcolor[rgb]{0.00,0.00,0.81}{#1}}
\newcommand{\BuiltInTok}[1]{#1}
\newcommand{\CharTok}[1]{\textcolor[rgb]{0.31,0.60,0.02}{#1}}
\newcommand{\CommentTok}[1]{\textcolor[rgb]{0.56,0.35,0.01}{\textit{#1}}}
\newcommand{\CommentVarTok}[1]{\textcolor[rgb]{0.56,0.35,0.01}{\textbf{\textit{#1}}}}
\newcommand{\ConstantTok}[1]{\textcolor[rgb]{0.56,0.35,0.01}{#1}}
\newcommand{\ControlFlowTok}[1]{\textcolor[rgb]{0.13,0.29,0.53}{\textbf{#1}}}
\newcommand{\DataTypeTok}[1]{\textcolor[rgb]{0.13,0.29,0.53}{#1}}
\newcommand{\DecValTok}[1]{\textcolor[rgb]{0.00,0.00,0.81}{#1}}
\newcommand{\DocumentationTok}[1]{\textcolor[rgb]{0.56,0.35,0.01}{\textbf{\textit{#1}}}}
\newcommand{\ErrorTok}[1]{\textcolor[rgb]{0.64,0.00,0.00}{\textbf{#1}}}
\newcommand{\ExtensionTok}[1]{#1}
\newcommand{\FloatTok}[1]{\textcolor[rgb]{0.00,0.00,0.81}{#1}}
\newcommand{\FunctionTok}[1]{\textcolor[rgb]{0.13,0.29,0.53}{\textbf{#1}}}
\newcommand{\ImportTok}[1]{#1}
\newcommand{\InformationTok}[1]{\textcolor[rgb]{0.56,0.35,0.01}{\textbf{\textit{#1}}}}
\newcommand{\KeywordTok}[1]{\textcolor[rgb]{0.13,0.29,0.53}{\textbf{#1}}}
\newcommand{\NormalTok}[1]{#1}
\newcommand{\OperatorTok}[1]{\textcolor[rgb]{0.81,0.36,0.00}{\textbf{#1}}}
\newcommand{\OtherTok}[1]{\textcolor[rgb]{0.56,0.35,0.01}{#1}}
\newcommand{\PreprocessorTok}[1]{\textcolor[rgb]{0.56,0.35,0.01}{\textit{#1}}}
\newcommand{\RegionMarkerTok}[1]{#1}
\newcommand{\SpecialCharTok}[1]{\textcolor[rgb]{0.81,0.36,0.00}{\textbf{#1}}}
\newcommand{\SpecialStringTok}[1]{\textcolor[rgb]{0.31,0.60,0.02}{#1}}
\newcommand{\StringTok}[1]{\textcolor[rgb]{0.31,0.60,0.02}{#1}}
\newcommand{\VariableTok}[1]{\textcolor[rgb]{0.00,0.00,0.00}{#1}}
\newcommand{\VerbatimStringTok}[1]{\textcolor[rgb]{0.31,0.60,0.02}{#1}}
\newcommand{\WarningTok}[1]{\textcolor[rgb]{0.56,0.35,0.01}{\textbf{\textit{#1}}}}
\usepackage{graphicx}
\makeatletter
\def\maxwidth{\ifdim\Gin@nat@width>\linewidth\linewidth\else\Gin@nat@width\fi}
\def\maxheight{\ifdim\Gin@nat@height>\textheight\textheight\else\Gin@nat@height\fi}
\makeatother
% Scale images if necessary, so that they will not overflow the page
% margins by default, and it is still possible to overwrite the defaults
% using explicit options in \includegraphics[width, height, ...]{}
\setkeys{Gin}{width=\maxwidth,height=\maxheight,keepaspectratio}
% Set default figure placement to htbp
\makeatletter
\def\fps@figure{htbp}
\makeatother
\setlength{\emergencystretch}{3em} % prevent overfull lines
\providecommand{\tightlist}{%
  \setlength{\itemsep}{0pt}\setlength{\parskip}{0pt}}
\setcounter{secnumdepth}{-\maxdimen} % remove section numbering
\ifLuaTeX
  \usepackage{selnolig}  % disable illegal ligatures
\fi
\IfFileExists{bookmark.sty}{\usepackage{bookmark}}{\usepackage{hyperref}}
\IfFileExists{xurl.sty}{\usepackage{xurl}}{} % add URL line breaks if available
\urlstyle{same}
\hypersetup{
  pdftitle={MTH 335 Exam 1},
  pdfauthor={Manny Fomundam},
  hidelinks,
  pdfcreator={LaTeX via pandoc}}

\title{MTH 335 Exam 1}
\author{Manny Fomundam}
\date{2023-10-08}

\begin{document}
\maketitle

\hypertarget{answer-the-following-questions.-please-answer-each-question-with-full-sentences-with-necessary-code-andor-graphics-to-support-your-answers.}{%
\subsection{Answer the following questions. Please answer each question
with full sentences with necessary code and/or graphics to support your
answers.}\label{answer-the-following-questions.-please-answer-each-question-with-full-sentences-with-necessary-code-andor-graphics-to-support-your-answers.}}

Question 1: ``Money can't buy happiness'' a) Load the data located on
D2L entitled income.data.csv.

\begin{Shaded}
\begin{Highlighting}[]
\FunctionTok{setwd}\NormalTok{(}\StringTok{"C:/Users/emman/OneDrive {-} Westminster College/Documents/MTH 335 01/Westminster College {-} MTH335/"}\NormalTok{)}
\NormalTok{data }\OtherTok{\textless{}{-}} \FunctionTok{read.csv}\NormalTok{(}\StringTok{"income.data.csv"}\NormalTok{)}
\FunctionTok{head}\NormalTok{(data)}
\end{Highlighting}
\end{Shaded}

\begin{verbatim}
##   X   income happiness
## 1 1 3.862647  2.314489
## 2 2 4.979381  3.433490
## 3 3 4.923957  4.599373
## 4 4 3.214372  2.791114
## 5 5 7.196409  5.596398
## 6 6 3.729643  2.458556
\end{verbatim}

\begin{enumerate}
\def\labelenumi{\alph{enumi})}
\setcounter{enumi}{1}
\tightlist
\item
  How would you describe the relationship between income and happiness?
\end{enumerate}

This shows that people who earn more typically have higher levels of
happiness. According to the data, a person's general happiness is
significantly influenced by their income.

\begin{Shaded}
\begin{Highlighting}[]
\CommentTok{\# Load the ggplot2 library}
\FunctionTok{library}\NormalTok{(ggplot2)}

\CommentTok{\# Create a scatter plot}
\FunctionTok{ggplot}\NormalTok{(data, }\FunctionTok{aes}\NormalTok{(}\AttributeTok{x =}\NormalTok{ income, }\AttributeTok{y =}\NormalTok{ happiness)) }\SpecialCharTok{+}
  \FunctionTok{geom\_point}\NormalTok{() }\SpecialCharTok{+}  \CommentTok{\# Add points for each data point}
  \FunctionTok{labs}\NormalTok{(}\AttributeTok{x =} \StringTok{"Income"}\NormalTok{, }\AttributeTok{y =} \StringTok{"Happiness"}\NormalTok{, }\AttributeTok{title =} \StringTok{"Money can\textquotesingle{}t buy happiness?"}\NormalTok{)}
\end{Highlighting}
\end{Shaded}

\includegraphics{Exam1_files/figure-latex/unnamed-chunk-2-1.pdf} c)
Compute the correlation coefficient for income and happiness. What does
this tell you about the relationship between these two variables?

The scatter-plot amply demonstrates the positive relationship between
happiness and money. Happiness levels are observed to rise along with
money.

\begin{Shaded}
\begin{Highlighting}[]
\CommentTok{\# Compute the correlation coefficient}
\NormalTok{correlation }\OtherTok{\textless{}{-}} \FunctionTok{cor}\NormalTok{(data}\SpecialCharTok{$}\NormalTok{income, data}\SpecialCharTok{$}\NormalTok{happiness)}

\CommentTok{\# Print the correlation coefficient}
\FunctionTok{print}\NormalTok{(correlation)}
\end{Highlighting}
\end{Shaded}

\begin{verbatim}
## [1] 0.8656337
\end{verbatim}

\begin{enumerate}
\def\labelenumi{\alph{enumi})}
\setcounter{enumi}{3}
\tightlist
\item
  Find the estimates for the ordinary least squares regression line (and
  give the full equation for the line) for happiness predicted by
  income. Superimpose the line on a scatterplot of the data.
\end{enumerate}

\begin{Shaded}
\begin{Highlighting}[]
\CommentTok{\# Fit the linear regression model}
\NormalTok{model }\OtherTok{\textless{}{-}} \FunctionTok{lm}\NormalTok{(happiness }\SpecialCharTok{\textasciitilde{}}\NormalTok{ income, }\AttributeTok{data =}\NormalTok{ data)}

\CommentTok{\# Extract the coefficients (intercept and slope)}
\NormalTok{intercept }\OtherTok{\textless{}{-}} \FunctionTok{coef}\NormalTok{(model)[}\DecValTok{1}\NormalTok{]}
\NormalTok{slope }\OtherTok{\textless{}{-}} \FunctionTok{coef}\NormalTok{(model)[}\DecValTok{2}\NormalTok{]}

\CommentTok{\# Create the equation of the regression line}
\NormalTok{equation }\OtherTok{\textless{}{-}} \FunctionTok{paste}\NormalTok{(}\StringTok{"Happiness ="}\NormalTok{, }\FunctionTok{round}\NormalTok{(intercept, }\DecValTok{2}\NormalTok{), }\StringTok{"+"}\NormalTok{, }\FunctionTok{round}\NormalTok{(slope, }\DecValTok{2}\NormalTok{), }\StringTok{"Income"}\NormalTok{)}

\CommentTok{\# Print the equation of the regression line}
\FunctionTok{print}\NormalTok{(}\StringTok{"Regression Equation:"}\NormalTok{)}
\end{Highlighting}
\end{Shaded}

\begin{verbatim}
## [1] "Regression Equation:"
\end{verbatim}

\begin{Shaded}
\begin{Highlighting}[]
\FunctionTok{print}\NormalTok{(equation)}
\end{Highlighting}
\end{Shaded}

\begin{verbatim}
## [1] "Happiness = 0.2 + 0.71 Income"
\end{verbatim}

\begin{Shaded}
\begin{Highlighting}[]
\CommentTok{\# Create a scatter plot with the regression line}
\FunctionTok{ggplot}\NormalTok{(data, }\FunctionTok{aes}\NormalTok{(}\AttributeTok{x =}\NormalTok{ income, }\AttributeTok{y =}\NormalTok{ happiness)) }\SpecialCharTok{+}
  \FunctionTok{geom\_point}\NormalTok{() }\SpecialCharTok{+}  \CommentTok{\# Add points for each data point}
  \FunctionTok{geom\_abline}\NormalTok{(}\AttributeTok{intercept =}\NormalTok{ intercept, }\AttributeTok{slope =}\NormalTok{ slope, }\AttributeTok{color =} \StringTok{"red"}\NormalTok{) }\SpecialCharTok{+}  \CommentTok{\# Add the regression line}
  \FunctionTok{labs}\NormalTok{(}\AttributeTok{x =} \StringTok{"Income"}\NormalTok{, }\AttributeTok{y =} \StringTok{"Happiness"}\NormalTok{, }\AttributeTok{title =} \StringTok{"Regression Line: Money can\textquotesingle{}t buy happiness?"}\NormalTok{)}
\end{Highlighting}
\end{Shaded}

\includegraphics{Exam1_files/figure-latex/unnamed-chunk-4-1.pdf}

\begin{enumerate}
\def\labelenumi{\alph{enumi})}
\setcounter{enumi}{4}
\tightlist
\item
  The median household income in New Wilmington, PA is about \$68,000.
  Use the line computed in question 4 to determine the happiness level
  predicted by the median household income of New Wilmington. Hint: you
  may have to change units. Use your prediction to compare to the data
  point in row 470. How close is this estimate?
\end{enumerate}

The predicted estimate (5.058284) from the actual estimate (4.985367)
actually very close as they have a difference of 0.07291729.

\begin{Shaded}
\begin{Highlighting}[]
\CommentTok{\# Median household income in New Wilmington, PA}
\NormalTok{median\_income }\OtherTok{\textless{}{-}} \FloatTok{6.8} \CommentTok{\#6.8 represents 68000}

\CommentTok{\# Predicted happiness using the regression line}
\NormalTok{predicted\_happiness }\OtherTok{\textless{}{-}}\NormalTok{ intercept }\SpecialCharTok{+}\NormalTok{ slope }\SpecialCharTok{*}\NormalTok{ median\_income}

\CommentTok{\# Print the predicted happiness}
\FunctionTok{print}\NormalTok{(}\StringTok{"Predicted Happiness for Median Income:"}\NormalTok{)}
\end{Highlighting}
\end{Shaded}

\begin{verbatim}
## [1] "Predicted Happiness for Median Income:"
\end{verbatim}

\begin{Shaded}
\begin{Highlighting}[]
\FunctionTok{print}\NormalTok{(predicted\_happiness)}
\end{Highlighting}
\end{Shaded}

\begin{verbatim}
## (Intercept) 
##    5.058284
\end{verbatim}

\begin{Shaded}
\begin{Highlighting}[]
\CommentTok{\# Extract happiness for the data point in row 470}
\NormalTok{actual\_happiness }\OtherTok{\textless{}{-}}\NormalTok{ data}\SpecialCharTok{$}\NormalTok{happiness[}\DecValTok{470}\NormalTok{]}

\CommentTok{\# Print the actual happiness}
\FunctionTok{print}\NormalTok{(}\StringTok{"Actual Happiness for Data Point in Row 470:"}\NormalTok{)}
\end{Highlighting}
\end{Shaded}

\begin{verbatim}
## [1] "Actual Happiness for Data Point in Row 470:"
\end{verbatim}

\begin{Shaded}
\begin{Highlighting}[]
\FunctionTok{print}\NormalTok{(actual\_happiness)}
\end{Highlighting}
\end{Shaded}

\begin{verbatim}
## [1] 4.985367
\end{verbatim}

\begin{Shaded}
\begin{Highlighting}[]
\CommentTok{\# Calculate the difference between predicted and actual happiness}
\NormalTok{difference }\OtherTok{\textless{}{-}} \FunctionTok{abs}\NormalTok{(predicted\_happiness }\SpecialCharTok{{-}}\NormalTok{ actual\_happiness)}

\CommentTok{\# Print the difference}
\FunctionTok{print}\NormalTok{(}\StringTok{"Difference between Predicted and Actual Happiness:"}\NormalTok{)}
\end{Highlighting}
\end{Shaded}

\begin{verbatim}
## [1] "Difference between Predicted and Actual Happiness:"
\end{verbatim}

\begin{Shaded}
\begin{Highlighting}[]
\FunctionTok{print}\NormalTok{(difference)}
\end{Highlighting}
\end{Shaded}

\begin{verbatim}
## (Intercept) 
##  0.07291729
\end{verbatim}

\begin{enumerate}
\def\labelenumi{\alph{enumi})}
\setcounter{enumi}{5}
\tightlist
\item
  Overall, would you say that ``money can't buy happiness''? I would say
  money does buy happiness based off this data.
\end{enumerate}

Question 2: Exploring data a) Load the diamonds data set from the
ggplot2 package. Describe each of the variables as numerical or
categorical. For numerical variables, describe each variable as
continuous or discrete, and for categorical variables, describe each
variable as ordinal or nominal.

carat Type: Numerical, Continuous cut Type: Categorical, Ordinal color
Type: Categorical, Ordinal clarity Type: Categorical, Ordinal depth
Type: Numerical, Continuous table Type: Numerical, Discrete price Type:
Numerical, Continuous x Type: Numerical, Continuous y Type: Numerical,
Continuous z Type: Numerical, Continuous

\begin{Shaded}
\begin{Highlighting}[]
\CommentTok{\#load the ggplot2 package}
\FunctionTok{library}\NormalTok{(ggplot2)}

\CommentTok{\#Load the "diamonds" dataset}
\FunctionTok{data}\NormalTok{(}\StringTok{"diamonds"}\NormalTok{)}
\end{Highlighting}
\end{Shaded}

\begin{enumerate}
\def\labelenumi{\alph{enumi})}
\setcounter{enumi}{1}
\tightlist
\item
  Calculate the mean and median carat of a diamond.
\end{enumerate}

\begin{Shaded}
\begin{Highlighting}[]
\CommentTok{\# Calculate the mean carat}
\NormalTok{mean\_carat }\OtherTok{\textless{}{-}} \FunctionTok{mean}\NormalTok{(diamonds}\SpecialCharTok{$}\NormalTok{carat)}

\CommentTok{\# Calculate the median carat}
\NormalTok{median\_carat }\OtherTok{\textless{}{-}} \FunctionTok{median}\NormalTok{(diamonds}\SpecialCharTok{$}\NormalTok{carat)}

\CommentTok{\# Print the mean and median carat}
\FunctionTok{print}\NormalTok{(}\StringTok{"Mean Carat:"}\NormalTok{)}
\end{Highlighting}
\end{Shaded}

\begin{verbatim}
## [1] "Mean Carat:"
\end{verbatim}

\begin{Shaded}
\begin{Highlighting}[]
\FunctionTok{print}\NormalTok{(mean\_carat)}
\end{Highlighting}
\end{Shaded}

\begin{verbatim}
## [1] 0.7979397
\end{verbatim}

\begin{Shaded}
\begin{Highlighting}[]
\FunctionTok{print}\NormalTok{(}\StringTok{"Median Carat:"}\NormalTok{)}
\end{Highlighting}
\end{Shaded}

\begin{verbatim}
## [1] "Median Carat:"
\end{verbatim}

\begin{Shaded}
\begin{Highlighting}[]
\FunctionTok{print}\NormalTok{(median\_carat)}
\end{Highlighting}
\end{Shaded}

\begin{verbatim}
## [1] 0.7
\end{verbatim}

\begin{enumerate}
\def\labelenumi{\alph{enumi})}
\setcounter{enumi}{2}
\tightlist
\item
  Suppose you want to determine the most popular cut of diamond. Create
  a graphic that looks at the cut vs.~count to determine the most
  popular cut.
\end{enumerate}

\begin{Shaded}
\begin{Highlighting}[]
\FunctionTok{library}\NormalTok{(ggplot2)}

\CommentTok{\# Create a bar plot of cut vs. count}
\FunctionTok{ggplot}\NormalTok{(diamonds, }\FunctionTok{aes}\NormalTok{(}\AttributeTok{x =}\NormalTok{ cut)) }\SpecialCharTok{+}
  \FunctionTok{geom\_bar}\NormalTok{() }\SpecialCharTok{+}
  \FunctionTok{labs}\NormalTok{(}\AttributeTok{x =} \StringTok{"Cut"}\NormalTok{, }\AttributeTok{y =} \StringTok{"Count"}\NormalTok{, }\AttributeTok{title =} \StringTok{"Count of Diamonds by Cut"}\NormalTok{)}
\end{Highlighting}
\end{Shaded}

\includegraphics{Exam1_files/figure-latex/unnamed-chunk-8-1.pdf}

\begin{enumerate}
\def\labelenumi{\alph{enumi})}
\setcounter{enumi}{3}
\tightlist
\item
  Within the cut of the diamond, suppose you want to know how clarity is
  related to each cut. Build onto your graphic from part (c) to include
  information about clarity in each cut.
\end{enumerate}

\begin{Shaded}
\begin{Highlighting}[]
\FunctionTok{library}\NormalTok{(ggplot2)}

\CommentTok{\# Create a grouped bar plot of cut vs. count, colored by clarity}
\FunctionTok{ggplot}\NormalTok{(diamonds, }\FunctionTok{aes}\NormalTok{(}\AttributeTok{x =}\NormalTok{ cut, }\AttributeTok{fill =}\NormalTok{ clarity)) }\SpecialCharTok{+}
  \FunctionTok{geom\_bar}\NormalTok{(}\AttributeTok{position =} \StringTok{"dodge"}\NormalTok{) }\SpecialCharTok{+}
  \FunctionTok{labs}\NormalTok{(}\AttributeTok{x =} \StringTok{"Cut"}\NormalTok{, }\AttributeTok{y =} \StringTok{"Count"}\NormalTok{, }\AttributeTok{title =} \StringTok{"Count of Diamonds by Cut and Clarity"}\NormalTok{) }\SpecialCharTok{+}
  \FunctionTok{theme}\NormalTok{(}\AttributeTok{axis.text.x =} \FunctionTok{element\_text}\NormalTok{(}\AttributeTok{angle =} \DecValTok{45}\NormalTok{, }\AttributeTok{hjust =} \DecValTok{1}\NormalTok{))}
\end{Highlighting}
\end{Shaded}

\includegraphics{Exam1_files/figure-latex/unnamed-chunk-9-1.pdf}

\begin{enumerate}
\def\labelenumi{\alph{enumi})}
\setcounter{enumi}{4}
\tightlist
\item
  Suppose you want to investigate which cut is the most popular for IF
  clarity (highest or best clarity). Create a visual to compare IF
  clarity in relation to cut. What conclusions can you make based on
  this visual?
\end{enumerate}

Based on the visual, ideal has the most number IF clarity which is the
best clarity which can be assumed that its a very popular cut for
consumers.

\begin{Shaded}
\begin{Highlighting}[]
\FunctionTok{library}\NormalTok{(ggplot2)}

\CommentTok{\# Filter the data for diamonds with IF clarity}
\NormalTok{diamonds\_IF }\OtherTok{\textless{}{-}}\NormalTok{ diamonds[diamonds}\SpecialCharTok{$}\NormalTok{clarity }\SpecialCharTok{==} \StringTok{"IF"}\NormalTok{, ]}

\CommentTok{\# Create a bar plot of cut vs. count for diamonds with IF clarity}
\FunctionTok{ggplot}\NormalTok{(diamonds\_IF, }\FunctionTok{aes}\NormalTok{(}\AttributeTok{x =}\NormalTok{ cut)) }\SpecialCharTok{+}
  \FunctionTok{geom\_bar}\NormalTok{() }\SpecialCharTok{+}
  \FunctionTok{labs}\NormalTok{(}\AttributeTok{x =} \StringTok{"Cut"}\NormalTok{, }\AttributeTok{y =} \StringTok{"Count"}\NormalTok{, }\AttributeTok{title =} \StringTok{"Count of Diamonds with IF Clarity by Cut"}\NormalTok{) }\SpecialCharTok{+}
  \FunctionTok{theme}\NormalTok{(}\AttributeTok{axis.text.x =} \FunctionTok{element\_text}\NormalTok{(}\AttributeTok{angle =} \DecValTok{45}\NormalTok{, }\AttributeTok{hjust =} \DecValTok{1}\NormalTok{))}
\end{Highlighting}
\end{Shaded}

\includegraphics{Exam1_files/figure-latex/unnamed-chunk-10-1.pdf}

\end{document}
